\documentclass[10pt]{article}

%% For including figures, graphicx.sty has been loaded in
%% elsarticle.cls. If you prefer to use the old commands
%% please give \usepackage{epsfig}
\usepackage{epsfig}

%% The amssymb package provides various useful mathematical symbols
%% The amsthm package provides extended theorem environments
\usepackage{amsmath,amsfonts,amssymb,amsthm}

%% The lineno packages adds line numbers. Start line numbering with
%% \begin{linenumbers}, end it with \end{linenumbers}. Or switch it on
%% for the whole article with \linenumbers.
\usepackage{lineno}

%% extra packages
\usepackage{kotex}
\usepackage{color,graphicx}
\usepackage{hyperref}
\usepackage{epsfig,fullpage}
\usepackage{natbib,cite}
\usepackage{url}
\usepackage{subfigure}
\usepackage{soul}
\usepackage{enumitem}
\usepackage{booktabs}
\usepackage{caption}
\usepackage{array,multirow}
\usepackage{stackrel}
\usepackage{appendix}
\usepackage{geometry}
\usepackage{amsmath}
\usepackage{mathtools}
\usepackage[onehalfspacing]{setspace}

% if you want to use upper case roman numerals, you have to
% define followings
\def\BigRoman{\uppercase\expandafter{\romannumeral\number\count 255 }}
\def\Romannumeral{\afterassignment\BigRoman\count255=}




\newtheorem{theorem}{Theorem}
\newtheorem{assumption}{Assumption}
\newtheorem{corollary}{Corollary}
\newtheorem{lemma}{Lemma}
\newtheorem*{remark}{Remark}
\newtheorem{proposition}{Proposition}


\begin{document}
\setstretch{1.5}

\title{대학원신입생세미나 과제1}

\author{통계학과 박현정 2015-20299} \date{\today}

\maketitle

\begin{abstract}
Probability Theory and Examples, Rick durret 교재의 $Theorem 1.6.8$을 정리. 
\end{abstract}

\section{Mesure theory}
\subsection{Expected Value} 


\begin{theorem}  \label{Definition}
Suppose $X_{N} \rightarrow X a.s.$ Let g, h be contimuos functions with
\begin{itemize}
\item[(i)] $g \geq 0 \; and \; g(x) \rightarrow \infty \; as\; |x|\rightarrow \infty$,
\item[(ii)] $|h(x)|/g(x) \rightarrow 0 \; as\; |x|\rightarrow \infty$
and \item[(iii)] $Eg(X_{n})\leq K < \infty \; $for all n.

\end{itemize}
Then $Eh(X_{n})\rightarrow Eh(X_{n})$.

\begin{proof}
By subtracting a contrast from $h$, we can suppose without loss of generality that $h(0)\;=\;0$. Pick $M$ large so that $P(|X|\;=\;M)\;0$ and $g(x)>0$ when $|x|\geq M$. Given a random variable Y, let $\bar{Y}=Y1_{(|y|\leq M)}$. Since $P(|X|\;=\;M)\;=\;0, \bar{X_{n}} \rightarrow \bar{X}$ a.s. Since $h(\bar{X_{n}}$ is bounded and h is countinous, it follows from the bounded convergence theorem that 

\begin{itemize}
\item[(a)]
$  Eh(\bar{X_{n}})\;\rightarrow\;Eh(\bar{X}) $
\item[(b)]
$ |Eh(\bar{Y})-Eh(Y)|\leq E|h(bar{Y})-h(Y)|\leq E(|h(Y)|;|Y|>M)\leq _{\in M}Eg(Y)$
where $_{\in M} \;=\;\sup\{|h(x)|/g(x) : |x| \geq M\}$.
To check the second inequality, not that when $|Y|\leq M, Y\;=\;Y$, and we have supposed $h(0)\;=\;0$. The third inequality follows from the definition of $_{\in m}$ .
\\
Taking $Y=X_{n}$ in (b) and using ({\romannumeral 3}), it follows that
\item[(c)] 
$|Eh(\bar{X_{n}})\;-\;Eh(X_{n})|\leq K_{\in M} $
\\
To estimate $|Eh(\bar{X_{n}})\;-\;Eh(X_{n})|$, we observe that $g\geq 0$ and $g$ is continuous, so Fatou's lemma implies
$$
Eg(X) \leq \liminf_{n \rightarrow \infty} Eg(X_{n}) \leq K
$$
Taking $Y\; = \;X$ in (b) gives
\item[(d)] 
$|Eh(\bar{X})\;-\;Eh(X)|\leq K_{\in M} $ \\
The triangle inequality implies 
\begin{eqnarray*}
|Eh(X_{n})\;-\;Eh(X)|  & \leq &  |Eh(X_{n})\;-\;Eh(X)|\\
&+&|Eh(\bar{X})\;-\;Eh(\bar{X})| + |Eh(\bar{X})\;-\;Eh(X)|
\end{eqnarray*}

Taking limits and using (a), (c), (d), we have
$$
\limsup_{n \rightarrow \infty} |Eh(X_{n}) \;-\; Eh(X)| \leq 2 K_{\in m}
$$
which proves the desired result since $K<\infty$ and 
$_{\in M} \;\rightarrow \; 0$ as $M \rightarrow \infty$.



\end{itemize}




\end{proof}

\end{theorem}




\end{document}



